\section{Analytical Usability Evaluation}
To make sure that our system would be usable in the end, two of our group members acted as experts to perform a Cognitive Walkthrough, while the third person additionally performed a KLM Analysis. We decided on these techniques, because in our opinion a Cognitive Walkthrough reveals most of the inefficencies and problems in the design of our website, while a KLM Analysis shows how well the webshop is actually designed.
The results of these evaluations are summed up in the following subsections. The complete results can be found in the appendix.

\subsection{Cognitive Walkthrough Setup}
To make sure that the results of the cognitive walkhtrough that our two experts conducted are comparable, we chose the following scenario as a starting point:

Susan, 82 years old, would like to buy trousers online. Her niece Pam gave birth to an adorable girl just three months ago. Since Susan is invited to go over to Pam's place for dinner, she would like to bring along a present for her. Pam's mentioned that her girl just grew out of another pair of trousers, so Susan would like to get her a new one. She would like to order a nice one, but it should't be too pricey. Susan knows that Pam likes yellow, which is why the trousers should be yellow if possible.

And these are additional assumptions:\\
Susan got her first computer two years ago, but she doesn't use it very often. She's ordered groceries online before, but she's never ordered anything from our shop. She doesn't like to create accounts and always pays with credit card, where possible.

\subsection{Results of the Cognitive Walkthrough}
Two of the more important problems that both experts found were:
\begin{itemize}\addtolength{\itemsep}{-0.5\baselineskip}
	\item User must have some familiarity with web shops to know what the cart is used for
	\item The "next step button" when choosing the payment method should be removed
\end{itemize}
The first problem could be addressed by providing a help feature that walks the user through the order process. This could be done with a simple textual description, or even with a small video that shows how to use the web shop. This would especially be helpful for first time users.
The "next step button" could easily be removed in a final system, by using JavaScript to advance the order process. However, a "back button" would have to be provided to return to the selection of the payment method, in case the user accidentally chooses the wrong method.

Another problem found has its roots in the usage of terminology. Although present in all the webshops we looked at, putting "Trousers" into the "Bottoms" category might not be evident to everyone. Especially for someone who has never used a clothes webshop before, this might be confusing. To solve this problem, the complete local navigation menu could be displayed at all times. Thus, all entries would be shown for all categories. However, this might not be optimal, since the menu then offers too many options at once. Better options would probably be, either to include a section about the terminology in a help feature, or to show all submenus as soon as the user's mouse is placed over a menu entry.

Most of the other problems that the two experts found are concerned with the ease of navigation and clarity, such as:
\begin{itemize}\addtolength{\itemsep}{-0.5\baselineskip}
	\item Cart doesn't show the number of items in it
	\item User shouldn't be able to click on "Add to cart" without selecting a color/size/quantity above zero
	\item It is not possible to remove an item from the cart
\end{itemize}
Of course, all of these problems would have to be addressed, if the system would actually be implemented. By using JavaScript, or adding a few buttons and drop-down lists to the design, these problems could easily be handled.

Overall, the design seems to be quite reasonable. Some mistakes were found, but none of them are very serious. However, only a user test with a final, implemented system could show, if the users can really navigate the system, as we intend it.

\subsection{Key-Level Model Analysis Setup}
We used the same scenario and the same assumptions for the KLM analysis that we already defined for the Cognitive Walkthrough. By doing this, the qualitative results of the Cognitive Walkthrough can be compared to the quantitative results of the KLM analysis. We also intend on using this scenario for the user test, because then the findings of the experts can be much easier compared to the outcomes of the user tests.

\subsection{Results of the Key-Level Model Analysis}
We chose to use this technique, to have a reference value, as how much time is needed to buy an item in our webshop. This will be especially useful for the planned user tests. There, we can easily compare the time a user needs to perform a task to the time our expert needed to perform the same task. With this, we can find out which steps may be more difficult for the future users and also find inefficencies in our system.\\
For the analysis, we identified the following actions that the users will have to perform:\\
\begin{tabular}{l l l}
	\hspace{0.5cm}- H: & Move hand to mouse & 0.4s\\
	\hspace{0.5cm}- M: & Mental preparation for following action & 9s\\
	\hspace{0.5cm}- P: & Point mouse to the next button & 0.2s\\
	\hspace{0.5cm}- F: & Fill in credit card details & 15s\\
\end{tabular}

Since we don't have an implemented system yet, we left away factors, such as the response time of the system. These factors might have a significant impact on the overall experience of using our webshop. However, since they mainly depend on factors such as used hardware, these times would have to be determined on the same system that we would use for the user testing.

Our expert found that 17 actions are necessary to buy a pair of girl trousers in our webshop. This is, if the user has no prior account and uses the guest checkout. The total time to perform these actions is 106.4 seconds. Of course, this does not include the decision process, exactly which pair the user wishes to buy. But since this may wary between a couple of seconds and several minutes, this is not a factor we can take into account.

All in all, we think this is a good result for our system. It seems to be straightforward to use and also efficiently designed.