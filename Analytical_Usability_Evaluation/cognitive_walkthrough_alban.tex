\documentclass[fontsize=12pt,paper=a4]{scrartcl}

\usepackage[utf8]{inputenc}
\usepackage[T1]{fontenc}
\usepackage{lmodern}

\usepackage[final]{microtype}

\usepackage{graphicx}

\begin{document}

\title{Evaluation}
\date{\today}
\author{Alban Edouard}
\maketitle


\section{Cognitive Walkthrough}

\subsection{Questions to ask}
\begin{description}
\item[Q1:] Will the user expect to have to take this action?
\item[Q2:] Will the user notice the control for the action?
\item[Q3:] Once users find the control, will they recognize that it produces the desired effect?
\item[Q4:] If the correct action is performed, will progress be apparent?
\end{description}
 
\subsection{Actions for buying a trouser for a girl}
\begin{table}[htdp]
\begin{center}
\begin{tabular}{|c|l|c|c|c|c|}
\hline
\textbf{Nr} & \textbf{Action} & \textbf{Q1} & \textbf{Q2} & \textbf{Q3} &\textbf{Q4} \\
\hline
1 & Click on ``Girl''& Yes & Yes & Yes & Yes\\
\hline
2 & Move Curser to ``Bottoms'' & No & Yes & Yes & Yes \\
\hline
3 & Move Curser to ``Trousers'' & Yes & Yes & Yes & Yes \\
\hline
4 & Click on ``Trousers'' & Yes & Yes & Yes & Yes \\
\hline
5 & Move to ``Order by: > Size'' & Yes & Yes & Yes & Yes \\
\hline
6 & Select ``0 - 6m'' from the drop down menu & Yes & Yes & Yes & Yes\\
\hline
7 & Move to ``Order by: > Prize'' & Yes & Yes & Yes & Yes \\
\hline
8 & Select ``10 - 20\pounds'' from the drop down menu & Yes & Yes & Yes & Yes\\
\hline
9 & Move to ``Order by: > Availability'' & Yes & Yes & Yes & Yes \\
\hline
10 & Select ``In stock'' from the drop down menu & Yes & Yes & Yes & Yes\\
\hline
11 & Click on ``Go'' & Yes & Yes & Yes & Yes\\
\hline
12 & Click on ``Girl Trousers 7'' & Yes & Yes & Yes & Yes\\
\hline
13 & Click on ``Choose Colour'' & Yes & Yes & Yes & Yes\\
\hline
14 & Select yellow from the colours & Yes & Yes & Yes & Yes\\
\hline
15 & Click on ``Choose Size'' & Yes & Yes & Yes & Yes\\
\hline
16 & Select ``3 months'' from sizes & Yes & Yes & Yes & Yes\\
\hline
17 & Click on ``Add to Cart'' & Yes & Yes & Yes & Yes\\
\hline
18 & Click on ``Cart'' & No & No & Yes & Yes\\
\hline
19 & Click on ``Checkout'' & Yes & Yes & Yes & Yes\\
\hline
20 & Click on ``Guest checkout'' & Yes & Yes & Yes & Yes\\
\hline
21 & Fill in shipping address & Yes & Yes & Yes & Yes\\
\hline
22 & Click on ``Next step'' & Yes & Yes & Yes & Yes\\
\hline
23 & Select ``Credit Card'' & Yes & Yes & Yes & Yes\\
\hline
24 & Click on ``Next step'' & Yes & Yes & No & Yes\\
\hline
25 & Fill in payment details & Yes & Yes & Yes & Yes\\
\hline
26 & Click on ``Next step'' & Yes & Yes & Yes & Yes\\
\hline
27 & Click on ``Place order'' & Yes & Yes & Yes & Yes\\
\hline
\end{tabular}
\end{center}
\label{Coginitive_walkthrough}
\caption{Coginitive walkthrough}
\end{table}%

\subsection{Problems found}
\begin{description}
\item[2 :] The user has to know that "Trousers" is in "Bottoms". 
\item[18 :] The user has to know that he has to clicked on the "Cart" button on the top right and corner to start the check out. A pop-up which indicate this action will be useful, with also an hyperlink in direction of the Cart.
\item[24 :] By clicking on "Next step", the user can only add more payment informations and not go in the final step.
\end{description}

\section{Heuristics Evaluation}

\subsection{Visibility of system status}

\subsection{Match between system and the real world}

\subsection{User control and freedom}

\begin{table}[htdp]
\begin{center}
\begin{tabular}{|p{2cm}|p{6.5cm}|p{6.5cm}|}
\hline
\textbf{Screen No(s)} & \textbf{What is wrong} & \textbf{How to improve?} \\
\hline
8 & It is not possible to choose the quantity for the item & Add a field where the user can choose the desired quantity.\\
\hline
9 & It is not possible to remove an item & Add a "Remove" button.\\
\hline
14 - Place an order & If the user has made any mistakes, it is not possible for him to correct it. & Add the following sentence next to the "Place order" button : You can modify your informations by clicking on one of the corresponding step, and modify your item by clicking on it.\\
\hline
\end{tabular}
\end{center}
\label{3_heurisitcs_eval}
\end{table}

\subsection{Consistency and standards}
\subsection{Consistency and standards}
\begin{table}[htdp]
\begin{center}
\begin{tabular}{|p{2cm}|p{6.5cm}|p{6.5cm}|}
\hline
\textbf{Screen No(s)} & \textbf{What is wrong} & \textbf{How to improve?} \\
\hline
12 & By clicking on "Next step", the user can only add more payment informations and not go in the final step. & Change the name of the button for "Payment Details".\\
\hline
\end{tabular}
\end{center}
\label{4_heurisitcs_eval}
\end{table}

\subsection{Error prevention}
\begin{table}[htdp]
\begin{center}
\begin{tabular}{|p{2cm}|p{6.5cm}|p{6.5cm}|}
\hline
\textbf{Screen No(s)} & \textbf{What is wrong} & \textbf{How to improve?} \\
\hline
8 & The user can select "Add to cart" even if any colors or sizes are chosen. & Enable the button only if the value is correct.\\
\hline
\end{tabular}
\end{center}
\label{5_heurisitcs_eval}
\end{table}

\subsection{Recognition rather than recall}

\subsection{Flexibility and efficiency of use}

\subsection{Aesthetic and minimalist design}

\subsection{Help users recognize, diagnose and recover from errors}

\subsection{Help and documentation}

\end{document}