\documentclass[fontsize=12pt,paper=a4]{scrartcl}

\usepackage[utf8]{inputenc}
\usepackage[T1]{fontenc}
\usepackage{lmodern}

\usepackage[final]{microtype}

\usepackage{graphicx}

\begin{document}

\title{Evaluation Expert 1}
\date{March 23, 2011}
\author{Joana Welti}
\maketitle

\section{Cognitive Walkthrough}

\subsection{Scenario}
Susan, 82 years old, would like to buy trousers online. Her niece Pam gave birth to an adorable girl just three months ago. Since Susan is invited to go over to Pam's place for dinner, she would like to bring along a present for her. Pam's mentioned that her girl just grew out of another pair of trousers, so Susan would like to get her a new one. She would like to order a nice one, but it should't be too pricey. Susan knows that Pam likes yellow, which is why the trousers should be yellow if possible. 

\subsection{Assumptions}
Susan got her first computer two years ago, but she doesn't use it very often. She's ordered groceries online before, but she's never ordered anything from our shop.
She doesn't like to create accounts and always pays with credit card, where possible.

\subsection{Questions to ask}
\begin{description}
\item[Q1:] Will the user expect to have to take this action?
\item[Q2:] Will the user notice the control for the action?
\item[Q3:] Once users find the control, will they recognize that it produces the desired effect?
\item[Q4:] If the correct action is performed, will progress be apparent?
\end{description}
 
\subsection{Actions for buying a trouser for a girl}
Cognitive walkthrough starts at the main page. The goal is to buy trousers for a girl.  
\begin{table}[htdp]
\begin{center}
\begin{tabular}{|c|l|c|c|c|c|}
\hline
\textbf{Nr} & \textbf{Action} & \textbf{Q1} & \textbf{Q2} & \textbf{Q3} &\textbf{Q4} \\
\hline
1 & Click on ``Girl''& Yes & Yes & Yes & Yes\\
\hline
2 & Move Curser to ``Bottoms'' & Maybe & Yes & Yes & Yes \\
\hline
3 & Move Curser to ``Trousers'' & Yes & Yes & Yes & Yes \\
\hline
4 & Click on ``Trousers'' & Yes & Yes & Yes & Yes \\
\hline
5 & Move to ``Order by'': > Prize'' & Yes & Yes & Yes & Yes \\
\hline
6 & Select ``< 10\pounds'' from the drop down menu & Yes & Yes & Yes & Yes\\
\hline
7 & Click on ``Go'' & Yes & Yes & Yes & Yes\\
\hline
8 & Click on ``Next ->'' go see the next overview page & Yes & Yes & Yes & Yes\\
\hline
9 & Click on ``Girl Trousers 10'' & Yes & Yes & Yes & Yes\\
\hline
10 & Click on ``Choose Colour'' & Yes & Yes & Yes & Yes\\
\hline
11 & Select yellow from the colours & Yes & Yes & Yes & Yes\\
\hline
12 & Click on ``Choose Size'' & Yes & Yes & Yes & Yes\\
\hline
13 & Select ``0 - 6m'' from sizes & Yes & Yes & Yes & Yes\\
\hline
14 & Click on ``Add to Cart'' & Yes & Yes & Yes & Yes\\
\hline
15 & Click on ``Cart'' & No & Yes & Yes & Yes\\
\hline
16 & Scroll down and click on ``Checkout'' & Yes & Maybe & Yes & Yes\\
\hline
17 & Click on ``Guest checkout'' & Yes & Yes & Yes & Yes\\
\hline
18 & Fill in shipping address & Yes & Yes & Yes & Yes\\
\hline
19 & Click on ``Next step'' & Yes & Yes & Yes & Yes\\
\hline
20 & Select ``Credit Card'' & Yes & Yes & Yes & Yes\\
\hline
21 & Click on ``Next step'' & Yes & Yes & Yes & No\\
\hline
22 & Fill in payment details & Yes & Yes & Yes & Yes\\
\hline
23 & Click on ``Place order'' & Yes & Yes & Yes & Yes\\
\hline
24 & Click on ``Next step'' & Yes & Yes & Yes & Yes\\
\hline
25 & Click on ``Go back to Home'' & Yes & Yes & Yes & Yes\\
\hline
\end{tabular}
\end{center}
\label{cog_walkthrough_girl_trouser}
\caption{Coginitive walkthrough for buying a girls trouser}
\end{table}%

\subsection{Problems found}
\begin{description}
\item[2.):] User can't just click on ``Bottoms'', he/she has wait until the system displays the next level of navigation and then move the curser to ``Trousers''. This behavior differs from the previous navigation. 
\item[15.):] For user who have used web shops before, it might be clear that the check out process starts by going to the cart. For less experienced users, this might not be inherently clear.
\item[16.):] If there are many items, the ``Checkout'' button is not visible without scrolling
\item[21.)] ``Next step'' doesn't open step 3, which would be the expected action, but more payment details. 
\end{description}

\section{Heuristics Evaluation}
\subsection{Visibility of system status}
Items can be ordered by availability, so that the user sees right away, which products are available at the moment. 


\subsection{Match between system and the real world}


\subsection{User control and freedom}
\begin{table}[htdp]
\begin{center}
\begin{tabular}{|p{2cm}|p{6.5cm}|p{6.5cm}|}
\hline
\textbf{Screen No(s)} & \textbf{What is wrong} & \textbf{How to improve?} \\
\hline
8 & There is no quantity selection. It is implicitly assumed that the user only wants one item. & Add an selection for the quantity as well, where ``1'' should be pre selected. \\
\hline
9 & Once items are in the cart, it should be easy to remove them again & Cart items should have a separate column with a delete option\\
\hline
\end{tabular}
\end{center}
\label{3_heurisitcs_eval}
\end{table}

Users can always go back to the main page by clicking on the logo on the top left side. Also, breadcrumbs are used throughout the system, which make it easy for the user to go back to a previous step. 

In the checkout process, it is possible to click on one of the previous steps to change information, where as it is not possible to go to a step that hasn't been filled out and sent to the system by clicking on ``Next step''.

\subsection{Consistency and standards}
\begin{table}[htdp]
\begin{center}
\begin{tabular}{|p{2cm}|p{6.5cm}|p{6.5cm}|}
\hline
\textbf{Screen No(s)} & \textbf{What is wrong} & \textbf{How to improve?} \\
\hline
5 & Navigation is not consistent. The categories for ``Girl'' are displayed on the left side, where the subcategories for ``Bottoms'' are only displayed when the user moves the mouse over ``Bottoms'' & When the user clicks on ``Bottoms'', display the subcategories on the left side, one level deeper than ``Bottoms'' \\ 
\hline
14 & There is no explicit ``Agree to Terms and Conditions'' box & Add a ``Agree to Terms and Conditions'' box, which the user must check before being able to place order\\
\hline
12 & Use of ``Next Step'' is misleading. Clicking on it doesn't open the next step, but displays more details about payment methods. & Use JavaScript to show further steps as soon as one payment method is selected \\
\hline
\end{tabular}
\end{center}
\label{4_heurisitcs_eval}
\end{table}


\subsection{Error prevention}
\begin{table}[htdp]
\begin{center}
\begin{tabular}{|p{2cm}|p{6.5cm}|p{6.5cm}|}
\hline
\textbf{Screen No(s)} & \textbf{What is wrong} & \textbf{How to improve?} \\
\hline
7 & It should't be possible to select multiple ``order by'' options (e.g size and prize) and then click on ``Go'', as it is unclear then what to use as ordering criterium. & Remove ``Go'' button, reorder items as soon as one criterium is selected (automatically, without having to click on ``Go'') or disable other selections as soon as one option is selected\\ 
\hline
8 & User shouldn't be able to click on ``Add to cart'' without selecting a color/size/quantity above zero. & Disable button with JavaScript, enable as soon as the user selects values for all of the them.\\
\hline
\end{tabular}
\end{center}
\label{5_heurisitcs_eval}
\end{table}


\subsection{Recognition rather than recall}
\begin{table}[htdp]
\begin{center}
\begin{tabular}{|p{2cm}|p{6.5cm}|p{6.5cm}|}
\hline
\textbf{Screen No(s)} & \textbf{What is wrong} & \textbf{How to improve?} \\
\hline
All & Cart doesn't show the number of items in it & Display the number of items so that the user sees if there is anything in it\\
\hline
\end{tabular}
\end{center}
\label{6_heurisitcs_eval}
\end{table}
Breadcrumbs show the exact location, starting from ``Home'', which help the user to remember in what step he/she is at.

\subsection{Flexibility and efficiency of use}
Creating an account can accelerate the checkout process, as the forms can be prefilled with the shipping address and the payment methods and details.

On screen 13, it is possible to use select ``Same as shipping address'', so that even users without an account don't have to enter their address twice. 

\subsection{Aesthetic and minimalist design}
\begin{table}[htdp]
\begin{center}
\begin{tabular}{|p{2cm}|p{6.5cm}|p{6.5cm}|}
\hline
\textbf{Screen No(s)} & \textbf{What is wrong} & \textbf{How to improve?} \\
\hline
6 & It is unclear where subnavigation of ``Bottoms'' disappears to once a subcategory of ``Bottoms'' is selected & When user clicks on ``Bottoms'', add another layer to the navigation with its subcategories \\
\hline
7 & Only ``Trousers'' is displayed, all the other subcategories are hidden. The user would have to move the mouse to ``Bottoms'' again in order to see all the options. & When user clicks on ``Bottoms'', add another layer to the navigation with its subcategories \\
\hline
\end{tabular}
\end{center}
\label{8_heurisitcs_eval}
\end{table}

\subsection{Help users recognize, diagnose and recover from errors}
So far, there are no screen mock ups with error messages. 

\subsection{Help and documentation}
\begin{table}[htdp]
\begin{center}
\begin{tabular}{|p{2cm}|p{6.5cm}|p{6.5cm}|}
\hline
\textbf{Screen No(s)} & \textbf{What is wrong} & \textbf{How to improve?} \\
\hline
8 & There is no size chart to help the user find the correct size & Add a link to a size chart \\
\hline
13 & No help boxes for ``Security number'', which might be unclear to first time users & Provide a help box with a picture of a credit card, so that it is clear to the user where to find this number \\
\hline
\end{tabular}
\end{center}
\label{10_heurisitcs_eval}
\end{table}

\end{document}