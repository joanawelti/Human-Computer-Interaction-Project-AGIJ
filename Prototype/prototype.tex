\section{Prototype}

For represented our prototype, we have designed a series of screen sketches contained in Powerpoint slides. Because we would like an easy to change (by designers or by users), to create and to simulate the prototype, we have decided to implement it in a low fidelity. The Powerpoint slides are in the appendix (slide 1 to 15).

\subsection{Find and select an item}
The first slide shows the home page of our website. By clicking on one of the buttons representing the global systems menus (on the top), the user can accessed to the corresponding local systems menus (as defined in the section \textit{6.3}) and thus access to the slides showing the items. Here, he has the possibility to sort the items using the "Order by" lists, and clicking and the "Go" button, and/or to select one of them. If there are many pages containing items, he can browse each of them thanks to the "Previous" and "Next" buttons (on the bottom). Use the "Search bar", and click on "Find", is another way for finding a specific item. \\

When the user selects an item, a new page appears with its description (price, picture, size, etc). After have added an item in his cart, "Add to cart" button, he can buy it.

\subsection{Buy an item}
When the user want to buy his items, he has to select the "Cart" button (as defined previously, be logged is not mandatory). Thus, he has an overview of each item and can choose its quantity, colour and size. He can then continue his shopping, "Continue shopping" button, or proceed to payment, "Checkout" button. \\

From now, we suppose that the customer doesn't want to be register and has selected "Checkout". The Checkout page is then shown (slide 10) and he has to select "Guest Checkout". From the step 1 to 3 of his Checkout, he must to fill his shipping and payment informations. The final step is to review and place his order. If everything is correct, the confirmation page is shown (slide 15).
