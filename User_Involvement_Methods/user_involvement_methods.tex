\section{User Involvement Methods}
In order to find out more about our future users, we decided to design a questionnaire and to do a contextual inquiry. 

\subsection{Questionnaire}
% Questionnaires are simple to administer and an easy way to get to know about the general attitude people have about buying clothes online, baby clothes in particular. Since we are building a web shop, our target group has access to a computer and an email account, so we designed our questionnaire in a way that it can be sent out by email. Our questions aim at finding out more about our users, their online shopping behavior as well as their previous experiences with web shops. 

We conducted our study in two phases. First, we designed a draft to do a pilot study to get some feedback from users. Then, we improved our questionnaire according to this feedback and did the actual study with the final version of the questionnaire.

\subsubsection{First Draft}
The questionnaire (figure \ref{fig:draft}, in the appendix) starts with an explanatory section on how to answer the multiple-choice questions, followed by some personal information inquiries, questions about web shops, online clothes stores in general and baby clothes stores in particular. It ends with a disclaimer on the usage of the questionnaire.

If you would like to know more about the purpose of the different sections, please refer to \ref{sec:explanations}.
The first draft was designed by three members of our group.

\subsubsection{Pilot study}
In order to test the questionnaire, we handed it out to the forth member of our group 
%to see what he thinks about it, being a computing science student and having attended the Human Computer Interaction Course as well. He also gave the questionnaire to seven other people.
and seven other people.

After the pilot tests, the questionnaire was updated with some improvements our team mate suggested. Also, some mistakes were corrected. Please refer to \ref{sec:improvements} for more information.
% In the introduction part, we added a simple question for the example in addition to the response. For the questions 3 and 12.a, instead of repeating the previous question, we just wrote "if yes", which is clearer. The fourth question was entirely rewritten because some people didn't understand the "computing expertise" term. In questions 6 and 10, "store" was replaced "shop" to keep the terminology consistent. The last update was the recasting of the question 11 and the inversion of its scale of values, from 1 = "not useful" to 5 = "very useful", because this seems to be more natural to people. 
The final version of our questionnaire is in the appendix (figure \ref{fig:final}).

\subsubsection{Results}
We got back 20 questionnaires from people between 19 and 62, with an average age of 35. 

Only half of all those people have children, of which three people have small children (nine months to 4 years). Almost everyone questioned spends at least some time on the computer every day and half of them use web shops two to six times a year, six people even more than once a month, and only one person has never bought anything from a web shop.

People buy a wide variety of items online, buy only eight people stated that they also buy clothes online. The majority of the people wrote that they don't mind to create an account, and people who do mind stated that they either don't use the shops often enough or that they have privacy concerns.
The credit cards seems to be the most popular means of paying, followed by the bank transfer.

We didn't get many answers concerning annoying features in web shops, but people did mention recommendations of other people or also user reviews (biased, not meaningful), and the fact that the sites normally don't remember where the user was at when returning from putting something into the shopping cart. When asking to rate the usefulness of different features, we learnt that this depends on the person. Every feature got 1s and 5s. Reviews of other people and the search functionality had the highest average (3.5 and 3.3), though. Clearly, most people know what a whishlist is, but hardly anyone ever uses it.

Most of those people who haven't bought any clothes online named a reason having to do with finding the correct size. Those who have shopped online wrote that they normally just buy the same size they would in a store and also quite a few stated that they check the size charts the sites provide (or should provide, according to our questioned people).
People mainly want to be able to filter for price, on sale or color. Gender and size seem to be less important. 

Unfortunately, only four people have bought baby clothes online. This seems to be something people do much less often than buying clothes for themselves. Three of these people bought clothes for a friend's child. Most people chose "age" as the criteria they would use to determine the correct size. Not so many people would like to have a second hand section.

\minisec{Implications}
One of our strongest findings is that not everyone can be pleased. As only the search functionality and the user review got marks above 3 on a scale from 1 to 5, we are only going to include those features. Some people have privacy concerns when having to creating an account, so we are going to offer guest checkout. As whishlist aren't used, we aren't going to include them. Finding the correct size for clothes is the main issue when buying online (even more so for babies), so we are going to provide a very detailed size chart with help texts and pictures. Although the second hand shop doesn't seem to be so popular, we decided that we want to include it anyway because we think that this would be a feature that distinguishes us from other companies. 

\subsection{Inquiry}
With our Contextual Inquiry, we wanted to find out what kind of problems people have when buying baby cloths online. 
Since our system is a greenfield engineering project, we didn't have a system at hand to let users work with. Therefore, we decided to use OshKosh B'Gosh~\footnote{http://www.oshkoshbgosh.com/}, a website we consider to be quite nice.
The inquiry consists of first questions, three tasks to perform by the inquired person, key features to look out for as an observer, and questions to ask afterwards.

If you want to learn more about the different parts, please refer to \ref{sec:inquiry_explanations} in the appendix.

The inquiry plan can be found in section~\ref{sec:contextual_inquiry}.

\subsubsection{Results}
We conducted three inquiries. One person had bought clothes for children online before. All of them have accounts for their favorite shops. When asked about what they like about these shops, we heard easy to use and clear navigation for example.

When our interviewees performed our first task, we found that every one chose a different strategy to find the item we asked for. Someone used the search bar and told us afterwards that with the help of searching, she found it very easy to complete the task. Another person used the navigation and then the price filter, and our last interviewee had quite a hard time, as he didn't see the search or filter functionality. He also didn't know which size to select, as he didn't understand the size chart. This person doesn't have any domain knowledge, where the other two have.

The second task was easy again when using the search bar. One person searched for ``white dress flowers'', which the system understood, so he found the dress right away. The only lady in our inquiry typed in ``baby girl dress'', which got her much more search results, so she used the size filer, which was the strategy of another person as well.

The third task was easy for everyone, so they all knew the concept of a shopping cart.

\minisec{Implications}
Different people use different ways to find an item, so we need to provide different possibilities. A good searching functionality and a very good indexing of our products seems to be a key point. Also, filters are used quite extensively. To encourage the usage of those features, we need to place them prominently and provide tooltips for inexperienced users. 
 
 


