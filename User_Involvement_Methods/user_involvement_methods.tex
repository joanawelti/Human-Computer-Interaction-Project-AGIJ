\section{User Involvement Methods}
In order to find out more about our future users, we decided to design a questionnaire and to do a contextual inquiry. 

\subsection{Questionnaire}
Questionnaires are simple to administer and an easy way to get to know about the general attitude people have about buying clothes online, baby clothes in particular. Since we are building a web shop, our target group has access to a computer and an email account, so we designed our questionnaire in a way that it can be sent out by email. Our questions aim at finding out more about our users, their online shopping behavior as well their previous experiences with web shops. 

We conducted our study in two faces. First, we designed a first draft to do a pilot study to get some feedback from users. Then, we improved our questionnaire according to this feedback and did the actual study with the final version of the questionnaire. To explain what the answers are used for, we wrote an accompanying email, which can be found in the appendix (...).

\subsubsection{First Draft}
The questionnaire starts with an explanatory section on how to fill out the questions, followed by some personal information inquiries, questions about web shops, online clothes stores in general and baby clothes stores in particular. It ends with a disclaimer on the usage of the questionnaire.

TODO: describe sections and why we ask those questions.

The first draft was implemented by three members of our group.

\subsubsection{Pilot study}
In order to test the questionnaire, we gave it to the forth member of our group to see what he thinks about it, being a computing science student and having attended the Human Computer Interaction Course as well.

TODO: Alban - describe how you conducted pilot study.

\subsubsection{Results}
We sent out .. questionnaires and got ... responses.
