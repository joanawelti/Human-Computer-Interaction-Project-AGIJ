\section{User Involvement Methods}
In order to find out more about our future users, we decided to design a questionnaire and to do a contextual inquiry. 

\subsection{Questionnaire}
Questionnaires are simple to administer and an easy way to get to know about the general attitude people have about buying clothes online, baby clothes in particular. Since we are building a web shop, our target group has access to a computer and an email account, so we designed our questionnaire in a way that it can be sent out by email. Our questions aim at finding out more about our users, their online shopping behavior as well as their previous experiences with web shops. 

We conducted our study in two phases. First, we designed a draft to do a pilot study to get some feedback from users. Then, we improved our questionnaire according to this feedback and did the actual study with the final version of the questionnaire. To explain what the questionnaire is all about, we wrote an accompanying email, which can be found in the appendix section ~\ref{sec:letter}

\subsubsection{First Draft}
The questionnaire (figure \ref{fig:draft} in appendix) starts with an explanatory section on how to answer the multiple-choice questions, followed by some personal information inquiries, questions about web shops, online clothes stores in general and baby clothes stores in particular. It ends with a disclaimer on the usage of the questionnaire.

The \textit{personal information} questions aim at finding out more about the person filling out the questionnaire. With these questions, we want to find with what kind of person (parent, age, experience etc.) we are dealing with. The \textit{web shops in general} section tries to find out more about the person's online shopping behavior. Especially, we would like to know if people mind to create an account when ordering items online, if wishlists are a feature that is actually used and what other features are valued most by users.
We also ask some more specific questions concerning clothes web shops to find out what the problems are when buying clothes online so that we can enhance our shop with features to make it easier to buy clothes online. 
Last, we have a few questions about \textit{baby clothes shops} in particular. We would like to include the feature that people can buy and sell used clothes online and we would like to know if users consider this useful. Also, we aren't sure how much help people need when deciding on the right size for baby clothes, so we have another question concerning selecting the correct size.

The first draft was designed by three members of our group.

\subsubsection{Pilot study}
In order to test the questionnaire, we handed it out to the forth member of our group to see what he thinks about it, being a computing science student and having attended the Human Computer Interaction Course as well. He also gave the questionnaire to seven other people.

After the pilot tests, the questionnaire was updated with some improvements our team mate suggested. Also, some mistakes were corrected. In the introduction part, we added a simple question for the example in addition to the response. For the questions 3 and 12.a, instead of repeating the previous question, we just wrote "if yes", which is clearer. The fourth question was entirely rewritten because some people didn't understand the "computing expertise" term. In questions 6 and 10, "store" was replaced "shop" to keep the terminology consistent. The last update was the recasting of the question 11 and the inversion of its scale of values, from 1 = "not useful" to 5 = "very useful", because this seems to be more natural to people. The final version of our questionnaire is in the appendix (figure \ref{fig:final}).

\subsubsection{Results}
We got back 20 questionnaires from people between 19 and 62, with an average age of 35. 

Unfortunately, we don't know that many people with children, so only half of all those people have children, of which three people have small children (nine months to 4 years). Almost everyone questioned spends at least some time on the computer every day and half of them use web shops two to six times a year, 6 people even more than once a month, and only one person has never bought anything from a web shop.

People buy a wide variety of items online, buy only eight people stated that they also buy clothes online. The majority of the people wrote that they don't mind to create an account, and people who do mind stated that they either don't use the shops often enough or that they have privacy concerns.
The credit cards seems to be the most popular means of paying (stated eleven times), followed by the bank transfer.

We didn't get many answers concerning annoying features in web shops, but people did mention recommendations of other people or also user reviews (biased, not meaningful), and the fact that the sites normally don't remember where the user was at when returning from putting something into the shopping cart. When asking to rate the usefulness of features, we learnt that it seems to be very hard to please everyone. Every feature got marks from reaching from 1 to 5. Review of other people and the search functionality had the highest average (3.5 and 3.3), though. Clearly, most people know what a whishlist is, but hardly anyone ever uses it (two people only). Most people have ordered clothes online before. Most of those who haven't named a reason having to do with finding the correct size. Those who have shopped online wrote that they normally just buy the same size they would in the store and also quite a few stated that they check the size charts the sites provide (or should provide, according to our questioned people).
People mainly want to be able to filter for price, on sale or color. Gender and size seem to be less important. 

Unfortunately, only four people have bought baby clothes online. This seems to be something people do much less often than buying clothes for themselves. Three of those people bought clothes for a friend's child. Most people chose "age" as the criteria they would use to determine the correct size, as this seems to be easier for people who don't buy clothes for their own kids. Not so many people would like to use a second hand functionality, but every one who's bought baby clothes online before had a "yes" for either or both of these questions. 

\minisec{Implications}
One of our strongest findings is that not everyone can be pleased. As only the search functionality and the user review got marks above 3 on a scale from 1 to 5, we are only going to include those features. Some people have privacy concerns when having to creating an account, so we are going to offer guest checkout. As whishlist aren't used, we aren't going to include them. Finding the correct size for clothes seems to be the main issue when buying online (even more so for babies), so we are going to provide a very detailed size chart with help texts and pictures. Although the second hand shop doesn't seem to be so popular, we decided that we want to include it anyway because we think that this would be a feature that distinguishes us from other companies. 

\subsection{Inquiry}
With our Contextual Inquiry, we wanted to find out what kind of problems people have when buying baby cloths online. Since our system is a greenfield engineering project, we didn't have a system at hand to let users work with. Therefore, we decided to use OshKosh B'Gosh~\footnote{http://www.oshkoshbgosh.com/}, a website we consider to be quite nice.
The inquiry consists of first questions, three tasks to perform by the inquired person, key features to look out for as an observer, and questions to ask afterwards.

The \textit{first questions} are designed to find out some background information about our inquired person in terms of their previous experiences with web shops. 
%Since we could't actually let them do the entire checkout process, we also asked a question about preferred payment method and creating accounts for shops. 

Our \textit{task descriptions} ask the inquired person to perform three different tasks. The first one has its focus on finding an item as cheap as possible, where the second one asks for a very specific item. The third task is based on the second one and was included to find out if our users are familiar with the shopping cart concept. 

While our inquired person does our tasks, we want to see how they find the products we ask them to find in order to learn more about how people navigate on a web shop. We also want to look for if the ``filter by'' and the search functionality is used to find out how we can include them in the most useful way in our project. 

The \textit{questions afterwards} focus on the experience the inquired person had.

The inquiry plan can be found in section~\ref{sec:contextual_inquiry}.

\subsubsection{Results}


