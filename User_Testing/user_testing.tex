\section{User Testing}

%TODO introduction

\subsection{What we want to find out in our test}
How fast can users perform certain tasks?
Tasks:
\begin{itemize}\addtolength{\itemsep}{-0.5\baselineskip}
	\item Buy a pair of yellow trousers for a girl of 6 months.
	\item Create an account for themself.
\end{itemize}
We would like to find out if an experienced user is faster than a beginner, in performing these tasks.

Additional questions:
\begin{itemize}\addtolength{\itemsep}{-0.5\baselineskip}
	\item Do the user find the right controls? How many mistakes do they make?
	\item Do the users like the design and layout of the webshop? And why?
\end{itemize}
We are employing the "summative evaluation" technique.


\subsection{The types of users we would like to use for our test}
To make sure that the test will give meaningful results, we would like to have at least 30 users with the following aspects, evenly distributed over our user groups:
\begin{itemize}\addtolength{\itemsep}{-0.5\baselineskip}
	\item Users who see a web shop for the first time
	\item Users who are experienced with buying things online
\end{itemize}
These are the groups of users we are looking for:
\begin{enumerate}\addtolength{\itemsep}{-0.5\baselineskip}
	\item Young user (around 20) with much computer experiences
	\item Middle age (around 40) with much and also with few experiences
	\item Older (around 60) with few experiences
\end{enumerate}
To find users within the first group, we would go to different universities, ask lecturers if it's possible to conduct a small pre-test with their students. To do this, we would create groups of people and for each group give them a short task, watch them and select the people that are most suitable for the test.
To find test subjects, for the second and third group we would like to use teachers and staff of different universities. And perform the same pre-test, as for the first group, to make sure that we have suitable candidates.
Finally, we would use the "matched pairs design" because it will be easy to match pairs within our test user groups. We also expect order effects to be significant, since more experienced users will be able to solve our tasks much more quickly.

Instructions for the test:\\
Just try to do the task and take your time. If you have any questions, please ask a member of the supervisors.

\subsection{Test location}
We will perform our test in a lab. It is easier to perform the test in a lab, than to go to people's homes and because we would like to use observational equipment. Also, if we provide the computer to run the test on, we make sure that everyone will use the same computer. Additionally, we don't want to put our website online, until it is really finished. Thus, it won't be possible to let people do the test at home.

\subsection{Observational methods}
We will both gather quantitative and qualitative data. The quantitative data, we wish to gather, is the time to complete our set task, the amount of errors the users make and the number of button presses, necessary to order an item in our webshop, or to create an account. This data can then be compared to our findings from the Key-Level-Mapping Analysis. The qualitative data will then tell us, where the problems lie and what problems the users have. To do this, we will make notes to accompany the video that we take while performing the test. These notes will include detail about possible confusions, hesitiaton, or mistakes of the users and the exact page and subtasks where they occured. This can then be compared to the Analytical Usability evaluation, perfomed by our experts.
We will encourage the users to think aloud and comment what they are doing. This will also give us an insight in the difficulties that the users have.

\subsection{Questionnaire for after the test}
%TODO 
Are you going to use a questionnaire for after the users have finished their tasks?
Yes


What questions will you ask? 
Have you enjoyed the test?
Have you had any problems?
Did you find all relevant information? If no, why?
Will you recommend this website to anyone else? And why?


