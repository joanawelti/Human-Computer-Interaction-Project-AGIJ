\section{Description of users and personas}
\subsection{Users and characteristics}

In our web application, there are mainly three types of users which are parents' grandparents and relatives and friends. We choose them as users because parents want to buy baby clothes to their babies' grandparents need to buy baby clothes to their grandchilds also relatives and friend prefer buying bay clothes as gifts to new parents.

\subsubsection{Parents}

Parents are between age of 16 and 50. They use shop more often than other users so they are frequent users. They have less domain experience particularly if they are new parents. They more need to buy baby clothes therefore they may prefer to buy cheaper clothes

\subsubsection{Grandparents}

Grandparents are between 50 and 99 They know about children in other words they have more domain experience. 
They do not use internet often so they need more help about the system. They tend to spend more money on baby clothes.

\subsubsection{Relatives and friends}

Relatives and friends are between 16 and 50. They do not know about children if they are not parents. They look for a gift so they tend to spend more money on baby clothes. They are not frequent users for this web shop, but they most likely have used web shops before.


\subsection{User Groups}
We identified three different user groups for our web shop, based on our own experiences with online shopping.
 
\subsubsection{Frequent users}
Frequent users use the web store on a regular basis. This user group mainly consists of parents, which need new clothes for their babies from time to time, as children grow out of their clothing quite fast in the first couple of years. Also, parents might need clothes for more than one child.

Frequent users know their way around the web site and are familiar with the checkout process. In order to speed up ordering, it is possible to create an account, which allows frequent users to save their shipping address and to review previous orders.

\subsubsection{First time/infrequent users}
First time or infrequent users don't know the web site yet, but have used web shops before. This group includes relatives, maybe also some grandparents, who are looking for a present for the parents and their baby. They might not want to register, so they should be provided the possibility to order without having to create an account.
 
\subsubsection{Users not accustomed to using web shops}
This user group mainly consists of grandparents, looking for clothes for their grandchildren. This user group doesn't use the internet often and needs extra assistance in ordering. Help/explanation texts and and easy and intuitive navigation are essential for this user group. 

\subsection{Personas}
\textbf{Jeff and Judy Seavers}
Aged 63 and 60, have just become grand parents for the first time, as their daughter Julie just gave birth to a little girl.
They have two children themselves. Jeff is a car mechanic while Judy learned dress making but stopped working just before they had their first child.
Both are capable of using a computer but don't use the Internet often, except for writing e-mails to their children.
They have never used an online shop before and thus need all the assistance they can get to navigate through the shop. For their picture, please refer to figure \ref{fig:seavers}.	

\textbf{Main idea}:
Grandparents, never used a web shop before, but with good domain knowledge. Not likely to become registered users, as they seldom buy baby clothes.

\textbf{Sarah Gordon}
Single mother, aged 28. Has had her first child two years ago and awaits her second. She is recently divorced and lives alone at the time being.
She is a teacher at the local high school and knows her way around computers. 
She enjoys spending as much time with her child. She only works 60\% and has to live on a tight budget. At least once a month she goes out with two of her best friends to hit the local clubs. Online shops are nothing new to her as she often shops online so she does not have to leave her child unattended or take him along to the store.
For her picture, please refer to figure \ref{fig:gordon}.	


\textbf{Main idea}:
Persona belonging to the parent category. Has good domain and also computer knowledge and is already a registered user.

\textbf{Bruce Walker}:
Bruce is a social worker living in London. He is 42 years old and is married, but without children.
His youngest sister Tracie has had a baby almost three years ago. He himself likes travelling too much to settle down and have children at the moment.
He seldom uses online shops, as he does not really like the idea of buying goods online. But he has used different webshops a few times. For his picture, please refer to figure \ref{fig:walker}.	


\textbf{Main idea}:
Middle-aged persona with some indirect domain knowledge. Some computer knowledge is present, but he is a new user to the site.

%\textbf{Greg Campbell}:
%He is studying Philosophy at the University of Aberdeen in his second year. Greg is 20 years old.
%His best friend Lisa, aged 22, just had her first child and he is really excited to buy some clothes for her baby.
%Greg met Lisa at the University, as she is studying Philosophy, too. He had a crush on her since freshers week but was too shy to ask her out.
%He often uses computers for his studies and regularly uses a couple of online shops, mainly to buy some new gadget.

%\textbf{Main idea}:
%A frequent web shop user, but new to this particular web shop. Has no domain knowledge but will probably register, due to his liking of online shops.